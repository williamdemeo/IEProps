%\documentclass[11pt]{amsart}
\documentclass{article}
\usepackage{amsmath,amssymb,mathrsfs}
\usepackage{latexsym,stmaryrd,enumerate,scalefnt}
\usepackage{acronym}
\newacro{flrp}[FLRP]{Finite Lattice Representation Problem}
\newacro{IE}{Interval Enforceable}
\newacro{cfIE}[cf-IE]{Core-free Interval Enforceable}
\newacro{minIE}[min-IE]{Minimal Interval Enforceable}
\newacro{CFSG}{Classification of Finite Simple Groups}


\newcommand{\tbeta}{\ensuremath{\widetilde{\beta}}}
\newcommand{\hbeta}{\ensuremath{\widehat{\beta}}}
\newcommand{\B}{{\mathbf{B}}}
\newcommand{\bB}{{\mathbf{B}}}
\newcommand{\sI}{\ensuremath{\mathcal{I}}}
\newcommand{\bA}{\ensuremath{\mathbf{A}}}
\newcommand{\Eq}{\ensuremath{\mathrm{Eq}}}
\newcommand{\sE}{\ensuremath{\mathcal{E}}}
\newcommand{\sO}{\ensuremath{\mathcal{O}}}
\newcommand{\sT}{\ensuremath{\mathscr{T}}}
\newcommand{\Palfy}{P\'alfy}
\newcommand{\Pudlak}{Pudl\'ak}
\newcommand{\lb}{\ensuremath{\llbracket}}
\newcommand{\rb}{\ensuremath{\rrbracket}}


\begin{document}
\noindent Author's response: ``Interval enforceable properties of finite groups''\\
\\
Manuscript Number: LAGB-2012-3903\\
\\
I would first like to thank the referee for many helpful comments and constructive
criticisms.  I believe the paper has been significantly improved by following
most of the referee's suggestions.\\
\\
Next I would like to address the referee's point that the paper would be
improved if we had an example of a collection $\{L_1, \dots, L_n\}$ of
lattices for which we believed:
\begin{itemize}
\item 
 there is no finite group $G$ having subgroups $H_i$ with $\lb H_i, G\rb
\cong L_i$, and 
\item proving there is no such $G$ will be easier than carrying out the programs
  described in other research.
\end{itemize}
I will try to make it clearer in the paper that I am not presenting such an
example.  Rather the paper presents a strategy which we hope will produce an
example collection that makes solving the problem trivial (because of the
parachute construction).  That is, we wish to identify lattices $\{L_i\}$ that force a
group to have certain properties $\{P_i\}$ that are collectively mutually exclusive.  

Perhaps there are certain collections of group properties for which it is
nontrivial to prove mutual exclusivity, but for the group properties I know of,
and for all those that have been proved interval enforceable, mutually
exclusivity is easy to check. 

One goal of the paper is to motivate the study of such properties.  Perhaps we
will eventually be able to prove that, for every collection $\{P_i\}$ of
\acs{cfIE} properties, there is a finte group having all of these properties.

\vskip1cm
\noindent {\bf Enumerated List of Suggestions.}
Below are the author's comments explaining how each point in the referee's
enumerated list of suggestions has been addressed.  (The page numbers refer
to the page numbers of the original manuscript, as used in the referee report.)
\begin{enumerate}[(1)]
% 1
\item The first two paragraphs of historical introduction have been
  shortened significantly, but not deleted entirely because the paper is not
  only about the \acs{flrp}.  It is also about the study of group properties
  linked to subgroup lattice structure, and this topic has a long history which
  I believe should be at least mentioned.
% 2
\item The suggested references are now included.
% 3
\item The definition of interval enforceable has been made clearer.
% 4
\item Lines 14--16 on page 4, along with the whole paragraph to which they
  belonged, have been deleted.
% 5
\item It should now be clearer that the fraktur symbol is being defined for the first
  time on this line.
% 6
\item On line 50 of page 6, ``the'' has been inserted between ``that'' and
  ``following.''
% 7
\item In footnote 7 on page 8, ``discuss'' has been changed to ``discussed.''
% 8
\item The ``tricky to follow'' argument is, in fact, a little bit tricky on the 
  first reading, but I am confident readers can follow it, perhaps with the
  assistance of a pencil.  I believe I have presented it in the simplest
  possible way, and I don't think it is too hard to follow, as long as
  one has a good picture of what is going on. Figure 2 is suggestive of the
  picture one should have in mind.
% 9
\item The expression $unu^{-1}$ has been replaced with $uwu^{-1}$.
% 10
\item I first learned about the notation $\lb U_0, U \rb_H$ from Borner's paper,
  and my intention was to conform with what appears elsewhere in the literature. I
  don't feel this change is crucial, and I would prefer to leave it as is.
  (In addition to my desire to conform with Borner's use, I'd rather not 
  have to change this is all of my notes, other papers, and slides about this
  topic.) 
% 11
\item The notation $\leq$ is standard. Given two algebras
  $\bA$ and $\bB$, be they lattices, groups, or what have you, the notation $\bA
  \leq \bB$ means $\bA$ is a subalgebra of $\bB$. Nonetheless, the referee's
  suggestion to make this explicit has been heeded in the remarks following
  Lemma 3.5. 
% 12
\item  
% 13
\item The two corollaries that now appear immediately after Dedekind's Theorem
  provide the essential fact about modular elements.
% 14
\item Footnote 13 on page 19 has been deleted.

\end{enumerate}

\vskip1cm
\noindent {\bf Other Changes.}
Some (not all) of the other changes I've made to the paper are as follows:
\begin{enumerate}[(1)]
\item 
The paragraph beginning on line 34 page 2 was deleted.
\end{enumerate}


\end{document}
